
\documentclass[12pt]{article}
%%%%%%%%%%%%%%%%%%%%%%%%%%%%%%%%%%%%%%%%%%%%%%%%%%%%%%%%%%%%%%%%%%%%%%%%%%%%%%%%%%%%%%%%%%%%%%%%%%%%%%%%%%%%%%%%%%%%%%%%%%%%%%%%%%%%%%%%%%%%%%%%%%%%%%%%%%%%%%%%%%%%%%%%%%%%%%%%%%%%%%%%%%%%%%%%%%%%%%%%%%%%%%%%%%%%%%%%%%%%%%%%%%%%%%%%%%%%%%%%%%%%%%%%%%%%
\usepackage{eurosym}
\usepackage{amssymb}
\usepackage{amsfonts}
\usepackage{amsmath}
\usepackage[nohead]{geometry}
\usepackage[singlespacing]{setspace}
\usepackage[bottom]{footmisc}
\usepackage{indentfirst}
\usepackage{endnotes}
\usepackage{graphicx}
\usepackage{rotating}
\usepackage{lscape}

\setcounter{MaxMatrixCols}{10}
%TCIDATA{OutputFilter=LATEX.DLL}
%TCIDATA{Version=5.00.0.2570}
%TCIDATA{<META NAME="SaveForMode" CONTENT="1">}
%TCIDATA{LastRevised=Monday, May 09, 2011 00:41:52}
%TCIDATA{<META NAME="GraphicsSave" CONTENT="32">}
%TCIDATA{Language=American English}

\newtheorem{theorem}{Theorem}
\newtheorem{acknowledgement}{Acknowledgement}
\newtheorem{algorithm}[theorem]{Algorithm}
\newtheorem{axiom}[theorem]{Axiom}
\newtheorem{case}[theorem]{Case}
\newtheorem{claim}[theorem]{Claim}
\newtheorem{conclusion}[theorem]{Conclusion}
\newtheorem{condition}[theorem]{Condition}
\newtheorem{conjecture}[theorem]{Conjecture}
\newtheorem{corollary}[theorem]{Corollary}
\newtheorem{criterion}[theorem]{Criterion}
\newtheorem{definition}[theorem]{Definition}
\newtheorem{example}[theorem]{Example}
\newtheorem{exercise}[theorem]{Exercise}
\newtheorem{lemma}[theorem]{Lemma}
\newtheorem{notation}[theorem]{Notation}
\newtheorem{problem}[theorem]{Problem}
\newtheorem{proposition}{Proposition}
\newtheorem{remark}[theorem]{Remark}
\newtheorem{solution}[theorem]{Solution}
\newtheorem{summary}[theorem]{Summary}
\newenvironment{proof}[1][Proof]{\noindent\textbf{#1.} }{\ \rule{0.5em}{0.5em}}
\makeatletter
\def\@biblabel#1{\hspace*{-\labelsep}}
\makeatother
\geometry{left=1in,right=1in,top=1.00in,bottom=1.0in}
\input{tcilatex}

\begin{document}

\title{Dynamic Programming with Discrete Choice: An Application to Asset
Price Crashes and Retirement Decisions}
\author{Rawley Z. Heimer\thanks{%
Corresponding Author: Rawley Z. Heimer, Brandeis International Business
School Mailstop 032, P.O. Box 549100, Waltham, MA 02454, USA. e-mail:
rheimer@brandeis.edu. } \\
%EndAName
{\normalsize International Business School, Brandeis University}}
\maketitle

\sloppy

\begin{center}
\textbf{Abstract}
\end{center}

%TCIMACRO{\TeXButton{noindent}{\noindent}}%
%BeginExpansion
\noindent%
%EndExpansion
This paper demonstrates how to use Matlab to solve a discrete time dynamic
programming problem that involves discrete choice. \ I present an
application in which I use a life-cycle model of consumption to analyze how
individuals make the decision to retire from work when faced with increased
tail risk in the distribution of asset prices. \ This research is relevant
considering the loss of wealth accumulated during the Great Recession. \
This tutorial is appropriate for a student studying intermediate to advanced
macroeconomics with some background in computer programming or statistical
packages. \ 

\strut 

\bigskip 

\bigskip 

\bigskip 

\bigskip 

\bigskip 

\bigskip 

\bigskip 

\bigskip 

\bigskip 

\bigskip 

\bigskip 

\bigskip 

\textbf{Keywords:} Asset price crashes; Consumption/saving; Dynamic
programming; Life-cycle model; Matlab; Retirement.

\strut

\textbf{JEL Classification Numbers:} F31, G11.

\pagebreak 
%TCIMACRO{\TeXButton{onehalfspacing}{\onehalfspacing}}%
%BeginExpansion
\onehalfspacing%
%EndExpansion

\section{Introduction}

According to the Federal Reserve, U.S. households lost \$5.1 trillion, or
nine percent, of accumulated wealth in the last quarter of 2008.\footnote{%
http://www.nytimes.com/2009/03/13/business/economy/13wealth.html} This
figure reflects dramatic declines in both stock prices and home values. \
Considering that many individuals rely on these assets to sustain
consumption during their retirement years, it is worth asking the question
how the retirement decisions of individuals has been affected by the recent
asset price crash. \ 

This research is most similar to that of Coile and Levine (2011) in that it
asks the question of how individuals have responded to the decline in asset
prices following the Great Recession. \ However, to address the question
empirically this work uses numerical methods and a structural model rather
than a reduced form specification. \ While the life-cycle model has been a
workhouse in economics for the past few decades, its foundation is most
often attributed to Friedman (1957). \ This research also uses the solution
method put into practice by Stock and Wise (1990). \ 

In the following section I present the model and its solution. \ In section
3, I demonstrate how to use Matlab to solve a model with discrete choice
using dynamic programming and conduct simulations. \ In the 4th section, I
provide exercises for the user to demonstrate their understanding of the
material and solutions.

\section{The Model}

Consider a life-cycle model in which a representative agent lives $T$
periods. \ She chooses how much physical goods to consume $c_{t}$, how much
leisure to pursue $l_{t}$ , and how much assets $a_{t+1}$ to accumulate each
period so as to maximize the expected value $E$ of lifetime utility $%
U(c_{t},l_{t})$:%
\begin{equation}
E_{t}\sum_{t=1}^{T}\beta ^{t}(U_{t}(c_{t},l_{t}))
\end{equation}

%TCIMACRO{\TeXButton{noindent}{\noindent}}%
%BeginExpansion
\noindent%
%EndExpansion
For the functional form of $U(c_{t},l_{t})$ I choose constant relative risk
aversion with non-separability between consumption and leisure:  
\begin{equation}
U_{t}(c_{t},l_{t})=\frac{(c_{t}^{\theta }l_{t}^{1-\theta })^{1-\sigma }-1}{%
1-\sigma }
\end{equation}

%TCIMACRO{\TeXButton{noindent}{\noindent}}%
%BeginExpansion
\noindent%
%EndExpansion
where $\beta $ is the subjective discount factor, $\theta $ is the
elasticity of consumption, and $\sigma $ coefficient of relative risk
aversion. \ The agent earns a constant income stream $y$ during the first $%
R<T$ periods of her life which she can either consume or save. \ Once the
agent retires, she receives retirement benefits $\lambda \in \lbrack 0,1]$
that are a fraction of her last paycheck starting in $t=R+1$ and lasting
until $T.$ \ In order to consume goods during retirement the agent also
earns a return on her saving ($1+r_{t}$). \ Thus, in any given period $1\leq
t\leq R$ the agent is constrained to the following law of motion:

\begin{equation}
a_{t+1}=(1+r_{t})a_{t}+y-c_{t}
\end{equation}

%TCIMACRO{\TeXButton{noindent}{\noindent}}%
%BeginExpansion
\noindent%
%EndExpansion
and in periods $R<t\leq T$:

\begin{equation}
a_{t+1}=(1+r_{t})a_{t}+\lambda y-c_{t}
\end{equation}

I further impose the restriction that the agent cannot be a net debtor in
any period (bankruptcy would be the preferred option)\ and that she exits
the life-cycle with zero wealth. \ These restrictions require $a_{t}\geq 0$
and $a_{T+1}=0$, respectively. \ 

The amount of leisure an individual consumes is simply one minus the amount
of work they supply, $n_{t}=1-l_{t}$. \ Furthermore, for $1\leq t\leq R$, $%
l_{t}$ is equal to a constant between zero and one, and, for $R<t\leq T$, $%
l_{t}=1$. \ These assumptions impose no flexibility over the number of hours
worked during their career (a notion that is not inconsistent with the
nature of many labor contracts) and, for simplicity, that retired
individuals do not take part time jobs. \ They are also unable to re-enter
the workforce after retirement.\ Considering the binary nature of leisure in
this model an analogous state variable would be for the agent to choose
their year of retirement, $R$.

The rate of return on assets $r_{t}$ is determined probabilistically each
period. \ I assume it evolves according to a three state Markov chain with a
transition matrix $\chi $:%
\begin{equation}
\chi (r^{h},r^{l},r^{c})=\Pr (r_{t}=r^{h}|r_{t}=r^{l},r_{t}=r^{h})
\end{equation}

%TCIMACRO{\TeXButton{noindent}{\noindent}}%
%BeginExpansion
\noindent%
%EndExpansion
for $r^{h},r^{l},r^{c}\in \{high,low,crash\}$ where $r^{h}>r^{l}>r^{c}$ and $%
r^{h},r^{l}\geq 0$, $r^{c}<0$. \ The former two states reflect normal
variation in the economy, boom and bust, while the latter is intended to
capture extreme events such as the Great Recession.

\subsection{The Solution to the Model}

It is not possible to find a closed form solution for any of the state
variables $c_{t},a_{t+1},l_{t},R$. \ \ Since this research is concerned with
the timing of retirement, I focus on developing a policy function, or
decision rule for $R$ when holding consumption constant. \ 

I can write the value function for the dynamic programming problem as the
discounted present value of lifetime utility and say that leisure is a
function of $R$:%
\begin{equation}
V_{t}(R)=\sum_{t=1}^{T}\beta ^{t}(U_{t}(l_{t}(R)))
\end{equation}

%TCIMACRO{\TeXButton{noindent}{\noindent}}%
%BeginExpansion
\noindent%
%EndExpansion
The value function can be rewritten by splitting up the summation into pre-
and post-retirement components:%
\begin{equation}
V_{t}(R)=\sum_{s=t}^{R-1}\beta ^{s-t}U_{w}(l_{s}(R))+\sum_{s=t}^{T}\beta
^{s-t}U_{R}(l_{s}(R))
\end{equation}

%TCIMACRO{\TeXButton{noindent}{\noindent}}%
%BeginExpansion
\noindent%
%EndExpansion
where $U_{w}$ is the utility while working and $U_{R}$ is the utility while
retired. \ 

Since the agent must choose whether to work during year $t$ implying $R>t$,
or retire so that $R=t$, she must compare the expected value of retirement
today with that which comes by retiring at any future date. \ Therefore, the
expected gain $G_{t}$ from delaying retirement until $R$ is found to be:%
\begin{equation}
G_{t}(R)=E_{t}V_{t}(R)-E_{t}V_{t}(t)
\end{equation}

I call the solution for the agent's retirement date $R^{\ast }$. \ 

\begin{equation}
R^{\ast }=\underset{R\in (t+1,t+2,...,T)}{\arg \max }E_{t}V_{t}(R)
\end{equation}

%TCIMACRO{\TeXButton{noindent}{\noindent}}%
%BeginExpansion
\noindent%
%EndExpansion
This result yields the following decision rule:

\begin{equation}
G_{t}(R^{\ast })=E_{t}V_{t}(R^{\ast })-E_{t}V_{t}(t)>0
\label{eq: Decision Rule}
\end{equation}

%TCIMACRO{\TeXButton{noindent}{\noindent}}%
%BeginExpansion
\noindent%
%EndExpansion
Equation \ref{eq: Decision Rule} states that the individual continues to
work at $t$ only if the gains from continuing to work remains positive. \ 

With this solution in hand, it is possible to use numerical methods to
determine when work is no longer preferred to retirement. \ I\ can calculate
the retirement date of the agent following some probabilistic sequence of
asset price returns. \ In the following section I demonstrate how to use
Matlab to derive the solution the problem. 

\section{Numerical Computing}

The following is a walkthrough of the script titled,
"LCretireOV\_3states\_sims.m". \ The work herein is an extension of
"consav.m" and calls the function "markov.m" (Hall, 2006). I proceed
sequentially through the script and refer to section titles provided within
the .m file.

\subsection{Set Parameter Values}

While many of the values in this section could have been included later in
the script file, for ease of use the parameters of the model are included in
the beginning. \ Doing so allows the researcher to easily conduct
comparative statics, or in other words, determine how changes in the
parameters of the problem effect the variables of interest.

The choice of parameter values, also called model calibration, is a
frequently contentious issue in economics. \ In this tutorial, parameters
are chosen to reflect the findings of micro-studies (for example, the
elasticity of consumption), government policies (for example, retirement
benefits being roughly a third of income), or historical averages found in
data (such as the percentage of the day spent working). \ 

\subsection{Form Wealth Grid}

The solution to the dynamic programming problem requires allowing the agent
to use backward induction from $T+1$ to create a set of complete contingent
outcomes. \ In other words, the agent must be able to arrive in any possible
state and choose how much assets to carry into the following period and when
to retire. \ In order to provide the agent with these options, I first
produce a grid for the agent's wealth. \ The agent is allowed to choose any
point on the grid which extends from zero to eighty in increments of 0.05. \
There is a trade-off between the precision of the grid-points and the size
of the grid, and the speed of computing.

\subsection{Calculate the Utility Function}

In this section, since the agent always chooses to consume all that is not
stored in the asset between periods, I first calculate how much consumption
the agent chooses in any of the three states both before and after
retirement. \ I then use these values to calculate the agent's utility in
each state. \ The lines of code 78-83 find all values in which consumption
falls below zero and replaces the corresponding utility with $-\infty $. \
Since the agent will never choose a utility of $-\infty $ it restricts the
agent to only positive values for consumption. \ 

\subsection{Use Backward Induction to Solve the Model}

In this section I calculate the agent's solution to the dynamic programming
problem. \ This requires the agent to choose the maximum value function in
each state. \ I calculate two four-dimension objects called "v" and
"tdecis". \ The former stores the maximum achievable value function in all
three states, in each period $t$, and for every possible retirement year $R$%
. \ The latter stores the corresponding grid-point index. 

\subsection{Use Backward Induction to Choose Retirement Date}

This segment of code is the primary innovation of this work. \ I show that
after calculating the complete set of contingent outcomes for all possible
retirement dates, I can then use the method and the decision rule described
in section 2.1 (equation 10) to calculate when the value of being retired
surpasses the value of continuing to work. \ This is achieved in line 146 by
determining which value function is greater, that with $R=t$ or $R=t+1$. \ I
store the results of all state contingent outcomes in the three-dimensional
object, "decRetire".

\subsection{Simulate Life of the Agent}

Lastly, I employ simulations of the agent's life-cycle to determine the
distribution of retirement dates. \ Lines 162-172 initiate the simulation by
specifying the number of simulations and other initial conditions as well as
generating matrices to store the results recorded during each pass. \ In
each simulation, the agent starts in period $t=1$ as an employed individual
living in the state with high asset price returns. \ During each period $t$%
,the agent first decides whether or not to retire and then determines how
much assets to carry into the following period. \ After each simulation all
state and control variables are recorded as well as the date in which the
individual chooses to retire in the vector "RetireYear". \ With this set of
results in hand I can then assess how the distribution of retirement years
is influenced by the distribution of asset prices. \ 

\section{Exercises and Answers}

\subsection{Compare the solution to the economy without asset price crashes}

Make a copy of the Matlab script "LCretireOV\_3states\_sims.m" and name it 
"LCretireOV\_2states\_sims.m". \ Modify the script so that there are only
two possible values for $r_{t}$, $r^{h},r^{l}\in \{high,low\}$. \ Pick
values for both $r^{h}$ and $r^{l}$, and a probability transition matrix $%
\chi $ such that the mean and variance are the same as that in the original
script with three states. \ Run the script and graph the resulting
distribution of retirement dates. \ Calculate the mean, variance, and skew
of the simulated retirement dates and compare the results in the two cases,
the model with and the model without the asset price crash. \ 

{\normalsize \singlespacing}%
%TCIMACRO{\TeXButton{noindent}{\noindent}}%
%BeginExpansion
\noindent%
%EndExpansion
\textbf{Solution: }\textit{See the attached Matlab file,
"LCretireOV\_2states\_sims.m". \ The results suggest that the mean
retirement year is increasing in the case where there are asset price
crashes. \ In the economy with only two states there is no variance or skew
in the retirement year. \ While this is a rare case, it does imply that the
presence of asset price crashes can increase both the variance and skew of
retirement years. \ }

{\normalsize 
%TCIMACRO{\TeXButton{onehalfspacing}{\onehalfspacing}}%
%BeginExpansion
\onehalfspacing%
%EndExpansion
}

\subsection{What happens if labor supply is allowed to vary?}

In the model above, leisure (equal to one minus labor supply) is constant
prior to retirement and equal to one thereafter. \ Consider the case in
which the agent is allowed to vary how much leisure they pursue (how much
labor they supply) each period while the decision to retire remains discrete
and irreversible. \ Without actually creating the script (you would likely
need stronger computing power to run it!), explain how to modify the .m file
to allow for this possibility. \ Predict what would happen to the variance
and skewness of retirement years when agents are allowed to adjust along
this margin.

{\normalsize \singlespacing}%
%TCIMACRO{\TeXButton{noindent}{\noindent}}%
%BeginExpansion
\noindent%
%EndExpansion
\textbf{Solution: }\textit{In order to allow for variable labor supply the
script would have to be modified to include another state variable. \ This
would require creating a second grid, this one being for labor supply, and
then adding another dimension to the calculated utility functions in lines
71-76 of "LCretireOV\_3states\_sims.m". \ When calculating the agent's
policy function for assets they would also need to decide what level of
labor supply maximizes their utility in every possible state. \ }

\textit{Allowing the agent to adjust along this margin would likely cause
them to vary their labor supply to respond to changes in asset price
returns. \ For example, when their asset prices fall, they will increase
their workload to avoid experiencing a large drop in consumption. \ Since
they can choose this option instead of delaying retirement to make up for
lost income its inclusion in the model will likely lead to lower variance
and reduced skewness in the distribution of retirement dates. \ }

{\normalsize \singlespacing}\pagebreak \nolinebreak 

\begin{thebibliography}{9}
\bibitem{CL2011} Coile, Courtney C. and Phillip B. Levine, \textit{"The
Market Crash and Mass Layoffs: How the Current Economic Crisis May Affect
Retirement,"} The B.E. Journal of Economic Analysis \& Policy: Vol. 11: Iss.
1 (Contributions), Article 22, 2011. 

\bibitem{FR1957} Friedman, Milton, \textit{"A Theory of the Consumption
Function,"} NBER Books, National Bureau of Economic Research, Inc, 1957,
number frie57-1.

\bibitem{HALL2006} Hall, George J. "Economics 303 Home Page." George J.
Hall's Homepage. Brandeis University, 30 Aug. 2006. Web. 09 May 2011. 
\TEXTsymbol{<}http://people.brandeis.edu/\symbol{126}ghall/econ303/%
\TEXTsymbol{>}.

\bibitem{SW1990} Stock, James H. and David A. Wise. \textit{"Pensions, the
Option Value of Work, and Retirement."} Econometrica, Vol. 58, No.5 (Sep.,
1990), pp. 1151-1180.
\end{thebibliography}

\end{document}
